\documentclass[11pt]{article}
\usepackage{fullpage}
\usepackage{amssymb}
\usepackage{enumitem}

\setlength{\parskip}{2ex}
\newcommand{\qedsymb}{\hfill{\rule{2mm}{2mm}}}
\newenvironment{proof}{\vspace{-1mm}\begin{trivlist}
}{\qedsymb\end{trivlist}\vspace{-1mm}}

\begin{document}
\begin{center}
{\bf \Large \bf CSC373 Winter 2015 Assignment \# 1}\\
\end{center}
{\bf Question 3} (Weidong An)\\
{\bf Algorithm}: Suppose there are $n$ CPOs(denoted by $c_1, ... , c_n$) and the examination periods last for $k$ days(denoted by $d_1, ..., d_k$).
\begin{itemize}

\item
    Let $P$ be the set of all examination periods and $P = \{d_1^m, ..., d_k^m\}\cup\{d_1^a, ..., d_k^a\}\cup\{d_1^e, ..., d_k^e\}$. $d_j^m$ indicates "the morning of day $d_j$". $d_j^a$ indicates "the afternoon of day $d_j$". $d_j^e$ indicates "the evening of day $d_j$".
\item
    Let $P_j$ be the set of all examination periods which are available for $c_j$ for $j = 1, ..., n$.
\item Based on $P_j$, create set $D_j$ which contains the dates on which $c_j$ is available for at least one examination period for $j = 1, ..., n$. Each element in $D_j$ has super script $j$. For example, if $c_3$ is available for days $d_1, d_3, d_5$, then $D_2 = \{d_1^2, d_3^2, d_5^2\}$.
\end{itemize}
Then, implement the following:
\begin{itemize}
\item[1.]
    Create a network flow $N$ with vertices $V=\{s, t\}\cup\{c_1, ..., c_n\}\cup P \cup (\bigcup_{i=1}^{n} D_i)$ and with edges
    \begin{itemize}
        \item[\textbullet] $E = (\{(s, c_1), ..., (s, c_n)\})$ (with $c(s, c_i) =$ maximum number of examinations that $c_i$ can
            invigilate)
        \item[\textbullet] $\cup(\bigcup_{i = 1}^{n}(\bigcup_{d\in D_i} \{(c_i, d)\}))$ (with $c(c_i, d) = 2$)
        \item[\textbullet] $\cup(\bigcup_{i=1}^{n}(\{(d, p)| d\in D_i, p\in P_i$ and $d, p$ have the same subscript(i.e. the same date)$\}))$ (with $c(d, p) = 1$)
        \item[\textbullet] $\cup(\bigcup_{p\in P} \{(p, t)\})$ (with $c(p, t) = \lceil$(number of examinations in period $p)\times (1+10\%)\rceil$)

    \end{itemize}
\item[2.]
    Find a maximum integer flow $f$ in network $N$ using Edmonds-Karp algorithm.
\item[3.]
    If there is an edge $(p, t)$ with $p \in P$ and $f(p, t) < c(p, t)$, return NIL. Otherwise, set $C_i = \{p|f(d, p)=1, d \in D_i, p\in P_i\}$ and return $C_1, ..., C_n$.
\end{itemize}
{\bf Runtime Analysis}:
\begin{itemize}
\item Notice that $|V| \leq nk+3k+2$. $|E|  \leq n + nk + 3nk+3k$.
\item Since Edmonds-Karp algorithm runs in $O(|V||E|^2)$, it takes $O((nk+3k+2)(n + nk + 3nk+3k)^2)=O(n^3k^3)$ to run Edmonds-Karp algorithm on $N$.
\item It takes $O(|V| + |E|)= O(nk)$ to build network $N$.
\item It takes $O(|E|)=O(nk)$ to build $C_i$ for $i = 1, ..., n$.
\item Totally, the algorithm runs in $O(n^3k^3)$ which is in polynomial time.
\end{itemize}
{\bf Justification of Correctness}
\begin{enumerate}
\item[Claim 1.] Every collection of valid sets of examination periods for CPOs $C_1, ..., C_n$ give rise to a flow $f$ in $N$.\\
                Since $C_1, ..., C_n$ are valid, we have the following:
                \begin{itemize}
                \item[(1)]
                    $f(s, c_i) = |C_i|$  (the number of examination periods that $c_i$ will invigilate)
                \item[(2)]
                    For $d\in D_i, f(c_i, d) =$ number of examination periods that $c_i$ will invigilate on day $d_i$ and it is no more than 2.
                \item[(3)]
                    For $d\in D_i, p\in P_i, f(d, p)=1$ if and only if $(d, p) \in C_i$
                \item[(4)]
                    For $p\in P, f(p, t) =$ number of CPOs in examination period $p = c(p, t)$ 
                \end{itemize}
                By (4), $|f|$ is maximized. Therefore, every valid collection of sets $C_1, ..., C_n$  gives rise of a maximum flow in $N$.
\item[Claim 2.] Every integer flow in $N$ gives rise to a collection of sets of examination periods for CPOs $C_1, ..., C_n$ (or NIL if it is not possible).
                \begin{itemize}
                \item
                    $C_i = \{p|f(d, p)=1, d \in D_i, p\in P_i\}$
                \item
                    Every CPO is within maximum availability because $c(s, c_i)=$ maximum number of examinations that $c_i$ can
            invigilate
                \item
                    Every CPO is assigned to no more than 2 examination periods in one day because $c(c_i, d) = 2, d\in D_i$.
                \item
                    Every CPO is only assigned to examination periods that is available because $(d, p) \notin E$ for $d\in D_i, p\notin P_i$.
                \item
                    Every examination period has enough CPOs if and only if $f(p, t) = c(p,t)$ for all $p\in P$. Therefore, $C_1, ..., C_n$ exist if and only if there is a flow such that $f(p, t) = c(p,t)$ for all $p\in P$. If such flow $f$ exists, $|f|$ must be maximized.
                    
                \end{itemize}
            Therefore, every maximized flow in $N$ gives rise to a collection of sets of examination periods for CPOs $C_1, ..., C_n$ if $f(p, t) = c(p,t)$ for all $p\in P$ otherwise NIL.
            

                
\end{enumerate}
By Claim 1 and Claim 2, the algorithm is correct.
\end{document}
