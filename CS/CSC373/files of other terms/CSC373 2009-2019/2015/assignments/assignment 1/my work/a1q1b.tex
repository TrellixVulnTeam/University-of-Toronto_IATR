\documentclass[a4paper,twoside,10pt]{report}

%% Language %%%%%%%%%%%%%%%%%%%%%%%%%%%%%%%%%%%%%%%%%%%%%%%%%
\usepackage[USenglish]{babel} %francais, polish, spanish, ...
\usepackage[T1]{fontenc}
\usepackage[ansinew]{inputenc}


\usepackage{lmodern} %Type1-font for non-english texts and characters
\usepackage{mathtools}
%% Packages for Graphics & Figures %%%%%%%%%%%%%%%%%%%%%%%%%%
\usepackage{graphicx} %%For loading graphic files
\usepackage{subfig} %%Subfigures inside a figure
\usepackage{geometry}

\usepackage{clrscode3e}

%\usepackage{pst-all} %%PSTricks - not useable with pdfLaTeX

%% Math Packages %%%%%%%%%%%%%%%%%%%%%%%%%%%%%%%%%%%%%%%%%%%%
\usepackage{amsmath}
\usepackage{amsthm}
\usepackage{amsfonts}
\usepackage{amssymb}


%% Page Margins etc %%%%%%%%%%%%%%%%%%%%%%%%%%%%%%%%%%%%%%%%%%
\setlength{\textheight}{24cm}
\setlength{\textwidth}{16cm}
\setlength{\topmargin}{-1cm}
\setlength{\oddsidemargin}{0pt}
\setlength{\evensidemargin}{0pt}

%%%%%% Misc %%%%%%
\newcommand{\blank}[1]{\hspace*{#1}\linebreak[0]}
\DeclarePairedDelimiter{\ceil}{\lceil}{\rceil}

\newcommand{\HRule}{\rule{\linewidth}{0.5mm}}
\newcommand{\abs}[1]{\lvert#1\lvert}
\newcommand{\marginline}{\noindent\makebox[\linewidth][r]{\rule{\textwidth}{1pt}}}
\newcommand{\exer}[1]{\noindent{\Large\textsc{Question #1}} \\ \marginline}
\newcommand{\exernoline}[1]{\textcolor{Dandelion}{\LARGE{Exercise #1}}}
\newcommand{\note}[1]{\textcolor{OrangeRed}{\bf{#1}}} % errors and/or other things to consider 
\newcommand{\lemma}[2]{\noindent \textit{\textbf{Lemma #1}}. #2}
\renewcommand{\qedsymbol}{\blacklozenge}


% New Environments %%%%%%%%%%%%%%%%%%%%%%%%%%%%%%%%%%%%%%%%
\newsavebox{\selvestebox}
\newenvironment{prob}
  {\newcommand\colboxcolor{FFFFFF}%
   \begin{lrbox}{\selvestebox}%
   \begin{minipage}{\dimexpr\columnwidth-2\fboxsep\relax}}
  {\end{minipage}\end{lrbox}%
   \begin{center}
   \colorbox[HTML]{\colboxcolor}{\usebox{\selvestebox}}
   \end{center}}

\newenvironment{proof*}{\begin{proof}[\textit{\textbf{Proof}}]}{\\ \end{proof}}
\newenvironment{soln}{\begin{proof}[\textit{\textbf{Solution}}]}{\\ \end{proof}}

\newenvironment{exercise}[1]{\exer{#1}}{}

%%%%%%%%%%%%%%%%%%%%%%%%%%%%%%%%%%%%%%%%%%%%%%%%%%%%%%%%%%%%

% \[\begin{tikzpicture}
% \draw[step=0.5cm,color=gray] (-1.5,-1.5) grid (1, 1);
% \begin{pgfonlayer}{myback}
% \fhighlight{m-1-1}{m-3-3}
% \fhighlight[green!30]{m-3-9}{m-5-13}
% \end{pgfonlayer}
% \end{tikzpicture}\]

%%%%%%%%%%%%%%%%%%%%%%%%%%%%%%%%%%%%%%%%%%%%%%%%%%%%%%%%%%%%
\begin{document}

\begin{exercise}{1b}
%\begin{soln}0
Consider the following algorithm:
\begin{codebox}
\Procname{\(\proc{Update-MST}(V, E, w, T, e_0, w_0)\)}
\li \If $e_{0} \in T$
    \Then
    \li \Comment Suppose $e_{0} = (u_{0}, v_{0})$
    \li \For $v \in V$
        \Do
        \li Augment $v$ to keep track of a representative vertex.
        \End
    \li Run DFS on $T - \{e_{0}\}$ starting from $u_{0}$, where for each encountered vertex $u$, set $u$'s representative to $u_{0}$
    \li Run DFS on $T - \{e_{0}\}$ starting from $v_{0}$, where for each encountered vertex $v$, set $v$'s representative to $v_{0}$
    \li best-weight = $\infty$
    \li best-node = NIL
    \li \For $e \in E - \{e_{0}\}$ 
        \Do
        \li \Comment suppose $e = (u, v)$
        \li \If representative($u$) $\neq$  representative($v$) and $w(e) <$ best-weight 
            \Then
            \li best-weight $\leftarrow w(e)$
            \li best-node $\leftarrow e$
            \End
        \End
    \li \If best-weight = $\infty$
        \Then
        \li \Return $\const{NIL}$
    \li \Else
        \li \Return $T - \{e_0\} \cup$ best-node
        \End
\li \Else
    \li  \Return $T$
    \End
\end{codebox}

\section{Runtime Analysis}
If the removed edge $e_{0}$ is not in $T$, then $T$ is still a valid MST for $G$, and the algorithm returns $T$ which is $O(1)$. Now let's consider the case where
$e_{0} \in T$.\\
\begin{enumerate}
\item Augmenting all vertices in $V$ to keep track of a
representative involves allocating $O(n)$ memory, which is takes $O(n)$ time.
\item Since $T$ is an MST, removing an edge $e_{0}$ from $T$ creates two disjoint trees, say $T_{1}$ and $T_{2}$, such that $|T_{1}| + |T_{2}| = T - 1$. Then running DFS on $T_{1}$ and $T_{2}$ and setting representatives (which is possible to implement in constant time) on lines 7-8 has a total runtime of $O(n)$.
\item Each loop of the for loop starting on line 9 takes a constant number of operations (accessing the representatives, comparing values, and setting values), which means that the entire for loop is $O(m)$
\end{enumerate}
Hence, the total worst case runtime is $O(n+m)$ which in practice is faster than $O(m\log^{*}n)$.


\section{Proof of Correctness}

We limit our attention to the case in which $e_0 \in T$. 

If there is no edge in $E$ which connects the disjoint trees created by the deletion of $e_0$ in $T$ ($V_1$ and $V_2$ in the algorithm), then removing $e_0$ also causes a disconnection in $G$ since by definition this means that the edges in $E$ but not in $T$ do not connect $V_1$ and $V_2$. In this case, it is impossible to construct a spanning tree, and so there is no solution, which is what is returned in the algorithm in this case (it returns NIL if it iterates through the edges and cannot find an edge connecting $V_1$ and $V_2$).

%Lemma 1: If there is no edge between the disconnection of %$T$ given by $(C_1, C_2)$ except $e_0$, then removing %$e_0$ from $E$ disconnects $G$.

Removing $e_0$ from $T$ yields two connected components $C_1$ and $C_2$, since $T$ was a minimum spanning tree (property). The algorithm simply finds the minimum cost edge $e$ between the two connected components, and adds this to $T - \{e_0\}$ as its result. Suppose, for the sake of contradiction, that the resulting tree $T_1 := T - \{e_0\} \cup \{e\}$ is not a minimum spanning tree. Then there must exist some other tree $T'$ (which is minimum) such that $w(T') < w(T)$. Consider the following cases:
\begin{enumerate}
\item \underline{Case 1: $e \notin T'$} \\
Then there must be some other edge $e'$ which connects a vertex from one of the edges in $C_1$, call it $u'$, to a vertex from one of the edges in $C_2$, call it $v'$ (otherwise there would be a disconnection and $T'$ would not be spanning). Now, by adding $e_{0}$ and removing $e'$ from $T'$, we obtain a tree $T'' \in E$ such that $w(T'') = w(T') - w(e') + w(e_0)$. We know that $w(T') = w(T) - w(e_0) + w(e)$ and, since based on the algorithm $e$ is cost edge connecting a vertex from one of the edges in $C_1$ to a vertex from one of the edges in $C_2$, we also know that $w(e) \leq w(e')$. Then $w(T'') = w(T) - w(e_0) + w(e) - w(e') + w(e_0)$ where $w(e) - w(e') \leq 0$, which means that $w(T'') \leq w(T)$, leading to a contradiction since $T$ is an MST for $G$.
\item \underline{Case 2: $e \in T'$} \\
Then at least one of the subtrees in $T'$ corresponding to $C_1$ and $C_2$ (i.e. the subtrees that contain the same vertices as in $C_1$ and $C_2$), denote them as $C_1', C_2'$ have a weight less than its counterpart. Suppose without loss of generality that $w(C_1)' < w(C_1)$. Then $w(C_1' \cup C_2 \cup e_0) < w(C_1 \cup C_2 \cup e_0) = w(T)$ and we have a contradiction since this gives a tree $T'' \subseteq E$ with a weight lower than $T$ which is an MST for $G$.
\end{enumerate}
Since either case leads to a contradiction, there does not exist a tree $T'$ where $w(T') < w(T)$. Hence, T is an MST.
%\end{soln}
\end{exercise}
\end{document}


