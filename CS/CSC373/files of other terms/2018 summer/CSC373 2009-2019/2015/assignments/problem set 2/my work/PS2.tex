\documentclass[11pt]{article}
\usepackage{fullpage}
\usepackage{amssymb}
\usepackage{enumitem}
\usepackage{clrscode3e}

\setlength{\parskip}{2ex}
\newcommand{\qedsymb}{\hfill{\rule{2mm}{2mm}}}
\newenvironment{proof}{\vspace{-1mm}\begin{trivlist}
}{\qedsymb\end{trivlist}\vspace{-1mm}}

\begin{document}
\begin{center}
{\bf \Large \bf CSC373 Winter 2015 Problem Set \# 2}\\
Name: Weidong An\\
Student Number: 1000385095\\
UTOR email: weidong.an@mail.utoronto.ca\\
\today\\
\end{center}

\begin{enumerate}[label=(\alph*)]

\item
No. Consider the following undirected graph $(G, w, L)$:
\begin{center}
$G.V = \{v_1, v_2, v_3\}$\\
$G.E = \{{e_1=(v_1, v_2), e_2=(v_1, v_3), e_3=(v_2, v_3)}\}$\\
$w(e_1)=w(e_2)=w(e_3)=1$\\
$L = G.V = \{v_1, v_2, v_3\}$
\end{center}

$G$ has 3 vertices, so the spanning trees of $G$ have 3 nodes. $L$ contains all the vertices of $G$.  Consider trees with 3 nodes. We can break them into two cases: the root has 2 children or the root has only 1 child. In the case that the root has 2 children, clearly the root can not be a leaf. In the case that the root has only 1 child, the child of the root must have 1 child, so this vertex can not be a leaf. Hence, in any spanning tree of $G$, there always exists a vertex which can not be a leaf.

%There are only three spanning trees of the graph $G$: the spanning tree rooted at $v_1$, the spanning tree rooted at $v_2$ and the spanning tree rooted at $v_3$. In any of the spanning trees, $v_1$ is not a leaf. Hence $(G, w, L)$ does not have a solution for this problem.

\item

The pseudocode are as follows. The idea is fixing Kruskal's algorithm. But instead of sorting the edges based their weights straightly, the algorithm defines a new weight function $w'$ and finds a subset $E'$ of $G.E$ first. The algorithm checks each edge in $G.E$. If neither of the end points is in $L$, add the edge to $E'$ and maintain the same weight. If one of the end points is in $L$ and the other is not in $L$, add the edge to $E'$ and increase the weight by $max$-$weight$ defined in line 1 (This is to make the weights of all this kind of edges greater than the edges in the first case). Otherwise, the edge would not be added into $E'$. Finally, call Kruskal's algorithm on $(V, E')$ and $w'.$
\begin{codebox}

\Procname{$\proc{MST-With-Set-Of-Leaves}(G, w, L)$}
\li $max$-$weight \gets max\{w(e)|$ for $e\in G.E\}$
\li $E' \gets \emptyset$
\li \For each edge $(u, v)\in G.E$
\li      \Then \If $u \notin L$ and $v \notin L$
             \Then
\li                   $w'((u, v)) \gets w((u, v))$
\li                   $E' \gets E' \cup \{(u, v)\}$
\li      \ElseIf $(u \in L$ and $v \notin L)$ or $(v \in L$ and $u \notin L)$
\li                 \Then $w'((u, v)) = w((u, v )) + max$-$weight$\li$E' \gets E' \cup \{(u, v)\}$\End
\End
\li       call Kruskal's algorithm on $(G'=(V, E'), w')$
\end{codebox}

\item
Comparing $E'$ with $E$, we find that $E'$ is obtained by cancelling all the edges connecting two vertices in $L$. Claim that the edges cancelled can not be in the required spanning trees. Suppose the edge $(u, v)$ where $u, v \in L$ is in a required spanning tree. Then, $u$ and $v$ are leaves of the spanning tree (i.e they should both be connected to only one vertex.). Since $u$ and $v$ are connected, $u$ and $v$ is not connected to other vertices contradicting the definition of spanning tree. (Or if $u$ and $v$ are the only two vertices in $G$, one of $u$ and $v$ cannot be a leaf.)\\
When Kruskal's algorithm is called, it sorts the edges into $e_1, ..., e_k, e_{k+1},...,  e_{m}$ where $e_1, ..., e_k $ connects two vertices in $ G.V - L$, $e_{k+1},..., e_{m} $connects two vertices one of which is in $ L$ and one of which is in $G.V - L$ and $w(e_1)\leq ... \leq w(e_k)$ and $w(e_{k+1})\leq ... \leq w(e_m)$ by the above algorithm. First, it finds a minimum spanning tree of $(G.V-L, \{e_1, ..., e_k\})$. At this point, vertices in $G.V - L$ are in the same connected component. Adding edges from $e_{k+1}, ..., e_{m}$ will make every vertex in $L$ a leaf. Kruskal's algorithm also ensures that edges selected from $e_{k+1}, ..., e_{m}$ has minimum total weight. Hence, the algorithm is correct.

\end{enumerate}

\end{document}
